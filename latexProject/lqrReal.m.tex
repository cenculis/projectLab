\normalsize\bf{Source code of linear quadratic regulator (Real implementation)}
\label{lqrReal.m}
\vspace{1cm}
\begin{lstlisting}[numbers=left,basicstyle=\scriptsize,caption={Source code of linear quadratic regulator (Real implementation).},captionpos=b]	
% Pause for 3 seconds before starting the code execution
pause(3);

% Clear workspace, close all figures, and clear command window
clear all;
close all;
clc;

% Add necessary paths for libraries
addpath(genpath('CLSS Praxis'), genpath('hudaqlib'))

% Hudaq device initialization
dev = HudaqDevice('MF634');

% Total experiment samples
s = 2000;
% Sampling period
Ts = 0.02;

% States during the experiment
x = zeros(s,4);
% Sample states
z2 = zeros(4,1);
% Initial values
x(1,:) = [(AIRead(dev,1)/0.15/100) 0.01*round(-13.1*AIRead(dev,3)) (-AIRead(dev,2)/0.96*pi/180) 0];

% Force array to store calculated force values
Force = zeros(s,1);

% Pendulum parameters
m_p     = 0.329; m_w     = 3.2; l_sp    = 0.44; f_w     = 6.2; 
f_p     = 0.009; gra       = 9.81; j_a     = 0.072; Ts = 0.02; 

% Continuous-time state-space matrices
A_c = [ 0   1                               0                   0
0   -f_w/(m_w+m_p)                  0                   0
0   0                               0                   1
0   (f_w*m_p*l_sp)/(j_a*(m_w+m_p)) (m_p*l_sp*gra)/j_a     -f_p/j_a];   
B_c = [0  ;   1/(m_w+m_p) ;   0   ;   -m_p*l_sp/((m_w+m_p)*j_a)];
C_c = [   1   0   0   0
0   1   0   0
0   0   1   0
0   0   0   1];
D_c = [0;0;0;0];

% Convert to discrete-time
sys_cont = ss(A_c,B_c,C_c,D_c);
sys_d = c2d(sys_cont,Ts);

A = sys_d.A; B = sys_d.B; C = sys_d.C; D = sys_d.D;

% LQR controller parameters
Q = diag([100, .1, 100, .1]);    % State penalization 
R = 1e-2;                        % Input penalization
Np = 60;                         % Prediction horizon 
[K, P] = dlqr(A, B, Q, R);

% Loop through each sample
for i = 1:s-1

% LQR CONTROL ALGORITHM
Force(i) = ctranspose(-K*x(i,:));

% Save the data
z2(1) = AIRead(dev,1)/0.15/100;     % Position of cart (meter)
z2(2) = 0.01*round(-13.1*AIRead(dev,3));    % Speed of cart (m/s)
z2(3) = -AIRead(dev,2)/0.96*pi/180;  % Angle of pendulum (radian)

% Voltage limitation
if abs(Force(i)) > 10
Force(i) = sign(Force(i))*10; 
end

% Apply calculated voltage    
tic
while toc < Ts
DOWriteBit(dev, 1, 2, 1);   % Activation pendulum
DOWriteBit(dev, 1, 2, 0);   % Channel 1 consists of DO0..DO7
DOWriteBit(dev, 1, 2, 1);   % DO2 Requires continuous pulse
AOWrite(dev, 2, Force(i));  % Apply calculated voltage
end

% Angular speed calculation (derivative) 
z2_winkel = -AIRead(dev,2)/0.96*pi/180;
z2(4) = (z2_winkel - z2(3))/Ts;   % Angular speed of pendulum

% Update states
x(i+1,:) = ctranspose(z2);

% Check if the pendulum is out of range
if abs(z2(1)) > 0.3 || abs(z2(3)*180/pi) > 10    
disp('Please bring me back !');
pause(3);   % Wait for 3 seconds
end
\end{lstlisting}
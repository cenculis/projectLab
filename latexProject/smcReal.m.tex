\normalsize\bf{Source code of sliding mode controller (Real implementation)}
\label{smcReal}
\vspace{1cm}
\begin{lstlisting}[numbers=left,basicstyle=\scriptsize,caption={Source code of sliding mode controller (Real implementation).},captionpos=b]	
pause(3);
clear all;
close all;
clc;

addpath(genpath('CLSS Praxis'),genpath('hudaqlib'))
dev = HudaqDevice('MF634');
mn = 25;
mkdir Messung_25;

s = 2000;  %total experiment sample
Ts = 0.02; % sampling period

x = zeros(s,4); %states during experiment
z2 = zeros(4,1);% sample states
x(1,:) = [(AIRead(dev,1)/0.15/100) 0.01*round(-13.1*AIRead(dev,3)) (-AIRead(dev,2)/0.96*pi/180) 0]; % inital values
Force = zeros(s,1); 

%% State-space
m_p     = 0.329;m_w     = 3.2;l_sp    = 0.44;f_w     = 6.2; 
f_p     = 0.009;gra       = 9.81;j_a     = 0.072;Ts = 0.02; 

A_c = [ 0   1                               0                   0
0   -f_w/(m_w+m_p)                  0                   0
0   0                               0                   1
0   (f_w*m_p*l_sp)/(j_a*(m_w+m_p)) (m_p*l_sp*gra)/j_a     -f_p/j_a];   
B_c = [0  ;   1/(m_w+m_p) ;   0   ;   -m_p*l_sp/((m_w+m_p)*j_a)];
C_c = [   1   0   0   0
0   1   0   0
0   0   1   0
0   0   0   1];
D_c = [0;0;0;0];

sys_cont = ss(A_c,B_c,C_c,D_c);
sys_d = c2d(sys_cont,Ts);

A = sys_d.A;B = sys_d.B;C = sys_d.C;D = sys_d.D;

%% Implementation
alpha = 6;  
beta = 1.8; % variable that changes the speed of switching resp the duty cycle
var_k = 10;

for i = 1:s-1

%  CONTROL ALGORITHM

s1 = - x(i,4) - alpha * x(i,3)- x(i,2)- beta * x(i,1); 
Force(i) = 12.335239*x(i,2)+68.877358*x(i,3)-3.492138*(0.125-alpha)*x(i,4)-3.492138*(1.7568-beta)*x(i,2)-(var_k * sign(s1));

% save the data
z2(1) = AIRead(dev,1)/0.15/100; % position of cart (meter)
z2(2) = 0.01*round(-13.1*AIRead(dev,3)); % speed of cart (m/s)
z2(3) = -AIRead(dev,2)/0.96*pi/180; % angle of pendulum (radian)

% voltage limitation
if abs(Force(i))>10
Force(i) = sign(Force(i))*10; 
end

% apply calculated voltage    
tic
while toc<Ts
DOWriteBit(dev,1,2,1)           % Freischaltung Pendel
DOWriteBit(dev,1,2,0)           % channel 1 besteht aus DO0..DO7
DOWriteBit(dev,1,2,1)           % DO2 requires continuous impuls
AOWrite(dev, 2, Force(i));      % apply calculated voltage
end

% angular speed calculation (derivative) 
z2_winkel = -AIRead(dev,2)/0.96*pi/180;
z2(4) = (z2_winkel-z2(3))/Ts; % angular speed of pendulum

x(i+1,:) = z2';

if abs(z2(1)) > 0.3 || abs(z2(3)*180/pi) > 10    % Der Pendel ist ausser Bereich
disp('Please bring me back !');
pause(3);                 % wait 3 second
end

end
\end{lstlisting}
\normalsize\bf{Source code of model predictive controller (Simulation)}
\label{mpcSim.m}
\vspace{1cm}
\begin{lstlisting}[numbers=left,basicstyle=\scriptsize,caption={Source code of model predictive controller (Simulation).},captionpos=b]	
% Clear workspace, command window, and import CasADi library
clear all;
clc

%% Parameters and Initialization
Ts = 0.02; % Sampling Time
time = 20; % Total simulation time

% Preallocate arrays for system variables
q   = zeros(2, time/Ts); % Position and angle
qd  = zeros(2, time/Ts); % Linear and angular velocity
qdd = zeros(2, time/Ts); % Linear and angular acceleration
q(:, 1) = [0.4; -0.1]; % Initial values for position and angle
tau   = zeros(1, time/Ts); % Voltage

q_r(1) = 0;  % Reference value of position
q_r(2) = 0;  % Reference value of angle
Q = diag([1, 1, 1e7, 1e2]);     % State penalisation 
R = 1e7;                        % Input penalisation
Np = 10;     % Prediction horizon 

qd_r = zeros(2, 1); % Reference value of linear and angular velocity

%% State-space
m_p     = 0.329; m_w     = 3.2; l_sp    = 0.44; f_w     = 6.2; 
f_p     = 0.009; gra       = 9.81; j_a     = 0.072; Ts = 0.02; 

% Continuous-time state-space matrices
A_c = [0   1                               0                   0
0   -f_w/(m_w+m_p)                  0                   0
0   0                               0                   1
0   (f_w*m_p*l_sp)/(j_a*(m_w+m_p)) (m_p*l_sp*gra)/j_a     -f_p/j_a];   
B_c = [0  ;   1/(m_w+m_p) ;   0   ;   -m_p*l_sp/((m_w+m_p)*j_a)];
C_c = [   1   0   0   0
0   1   0   0
0   0   1   0
0   0   0   1];
D_c = [0;0;0;0];

% Convert to discrete-time
sys_cont = ss(A_c,B_c,C_c,D_c);
sys_d = c2d(sys_cont,Ts);

A = sys_d.A; B = sys_d.B; C = sys_d.C; D = sys_d.D;
[K, P] = dlqr(A, B, Q, R);	% Linear quadratic regulator for callculating the P matrix

%% Pendulum parameters
KF=2.6; M0=3.2; M1=0.329; M=M0+M1; ls=0.44; inert=0.072; N_val=0.1446;
N01_sq=0.23315; Fr=6.2; C=0.009; gra=9.81;

a32 = -N_val^2/N01_sq*gra ; a33 = -inert*Fr/N01_sq; a34 = N_val*C/N01_sq; 
a35 = inert*N_val/N01_sq; a42 = M*N_val*gra/N01_sq; a43 = N_val*Fr/N01_sq; a44 = -M*C/N01_sq;
a45 = -N_val^2/N01_sq; b3=inert/N01_sq; b4=-N_val/N01_sq;
b3_hat = inert/N01_sq+0.1; b4_hat = -N_val/N01_sq+0.1;

%% Animation parameters
xmin = -1;
xmax = +1;

figure;
h    = [];
h(1) = subplot(4,2,1);
h(2) = subplot(4,2,3);
h(3) = subplot(4,2,5);
h(4) = subplot(4,2,7);
h(5) = subplot(2,2,2);
h(6) = subplot(2,2,4);

Disturbance = 1;

% Hessian, F matrices of cost function
[H, F] = Controller_MPC_CostFunction(A, B, Np, Q, R, P);
s = zeros(1, Np);
s(1, 1) = 1;

for k = 1:time/Ts-1

% CONTROL ALGORITHM - Explicit MPC
tau(1, k) = -s*(H^(-1))*F*[q(1, k); qd(1, k); q(2, k); qd(2, k)];

% Voltage limitation
if abs(tau(:, k)) > 10
tau(:, k) = sign(tau(:, k)) * 10;
end

% Inverted pendulum math model
beta_x2 = (1 + N_val^2/N01_sq*(sin(q(2, k)))^2)^(-1);
qdd(:, k+1) = [beta_x2*(a32*sin(q(2, k))*cos(q(2, k))+a33*qd(1, k)+...
a34*cos(q(2, k))*(qd(2, k))+a35*sin(q(2, k))*qd(2, k)^2+b3*tau(:, k));
beta_x2*(a42*sin(q(2, k))+a43*cos(q(2, k))*qd(1, k)+...
a44*(qd(2, k))+a45*cos(q(2, k))*sin(q(2, k))*(qd(2, k))^2+b4*cos(q(2, k))*tau(:, k))];

qd(:, k+1) = qd(:, k) + qdd(:, k+1)*Ts;        
q(:, k+1) = q(:, k) + qd(:, k+1)*Ts;
q(2, k+1) = mod(q(2, k+1) + pi, 2*pi) - pi;

% Disturbance
if mod(k, 60) == 0 && Disturbance == 1
x = rand();
if x > 0.5 && k ~= 200
q(2, k+1) = q(2, k+1) + rand()*0.12; 
else
q(2, k+1) = q(2, k+1) - rand()*0.12; 
end
if k == 200
q(2, k+1) = q(2, k+1) - 0.1;  
end
end

% Plotting animation
plot(0, 'Parent', h(6));
hold on;
p1 = -q(1, k);
p2 = -q(1, k) + ls*exp(1i*(q(2, k)+pi/2));
line(real([p1, p2]), imag([p1, p2]));
plot(real(p2), imag(p2), '.', 'markersize', 40);
hold off;

% Center plot w.r.t. object
if q(1) > xmax
xmin = xmin + 0.1;
xmax = xmax + 0.1;
elseif q(1) < xmin
xmin = xmin - 0.1;
xmax = xmax - 0.1;
end

grid on;
axis([h(1)], [0 time/Ts-1 -1 1]); title('position')
axis([h(2)], [0 time/Ts-1 -0.5 0.5]); 
axis([h(3)], [0 time/Ts-1 -1 1]); 
axis([h(4)], [0 time/Ts-1 -5 5]); 
axis([h(5)], [0 time/Ts-1 -10 10]); 
axis([h(6)], [xmin-.2 xmax+.2 -.5 .5]);

% Update animation
drawnow;

% Plotting system variables
plot(q(1, 1:k), 'b', 'Parent', h(1)); % Position 
plot(q(2, 1:k), 'b', 'Parent', h(2)); % Angle 
plot (qd(1, 1:k), 'b', 'Parent', h(3)); % Velocity
plot(qdd(1, 1:k), 'b', 'Parent', h(4)); % Angular velocity
plot(tau(1:k), 'k', 'Parent', h(5)); % Torque

title(h(1), 'position');
title(h(2), 'angle');
title(h(3), 'velocity');
title(h(4), 'angular velocity');
title(h(5), 'torque');
end
\end{lstlisting}